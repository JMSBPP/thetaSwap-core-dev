\documentclass[11pt]{article}
\usepackage[margin=1in]{geometry}
\usepackage[T1]{fontenc}
\usepackage[utf8]{inputenc}
\usepackage{lmodern}
\usepackage{amsmath}
\usepackage{enumitem}
\setlist[itemize]{noitemsep, topsep=2pt}
\setlist[enumerate]{noitemsep, topsep=2pt}

\title{ThetaSwap Business Model}
\author{}
\date{\today}

\begin{document}
\maketitle
\tableofcontents

\section{Introduction}
ThetaSwap aims to standardize the time/range (theta) component of concentrated liquidity provision and make it explicitly tradable and hedgeable.

The working thesis from the draft notes is:
\begin{equation}
\text{LP payoff} = P_T - (P_T-K)^+ + \text{RangePayoff}
\end{equation}
and equivalently
\begin{equation}
\text{RangePayoff} = \text{LP} - P_T + (P_T-K)^+.
\end{equation}

This motivates a new structured instrument that isolates the non-covered-call component of LPing and turns it into a standardized market.

\subsection*{Problem Statement}
\begin{itemize}
  \item LPs face adverse selection and LVR, and often cannot isolate their range/theta exposure.
  \item Existing LP overlays are fragmented and hard to compose.
  \item JIT behavior creates externalities that are not directly priced.
\end{itemize}

\subsection*{Product Objective}
\begin{itemize}
  \item Tokenize and price range/theta exposure from LP positions.
  \item Enable LP hedging and structured speculation.
  \item Create a market signal for time-in-range value and JIT externalities.
\end{itemize}

\section{Instrument Design}
\subsection{Range Accrual Note (RAN)}
ThetaSwap introduces a \textbf{Range Accrual Note (RAN)} linked to a concentrated LP position.

\textbf{Core terms (draft):}
\begin{itemize}
  \item Underlying: \texttt{feeGrowthInside(liquidityPosition)}
  \item Strike interval: tick range $[i_{low}, i_{up}]$
  \item Observation schedule: block-based or epoch-based
  \item Coupon rule: accrues when spot is inside strike interval
\end{itemize}

\textbf{Coupon intuition (from notes):}
\[
\text{coupon}(t)=
\begin{cases}
\text{fee} \times \text{swapFlow}(t), & \text{if } P_t \in [i_{low}, i_{up}]\\
0, & \text{otherwise}
\end{cases}
\]

\subsection{Tokenization and Positions}
\begin{itemize}
  \item Long RAN token: receives range accrual cash flows.
  \item Short RAN token: finances/pays range accrual cash flows.
  \item Minting can be linked to an active LP position (or a verified LP vault share).
\end{itemize}

\subsection{Product Family Extensions}
Potential extensions already identified in notes:
\begin{itemize}
  \item Dynamic accrual frequency updates
  \item Target redemption notes
  \item Knock-in / knock-out barriers
  \item Callable variants
  \item Basket underliers (multi-asset vaults)
\end{itemize}

\section{Pricing Model}
\subsection{No-Arbitrage Decomposition}
RAN valuation follows the decomposition logic in the draft:
\[
\text{LP} + \text{short perpetual} + \text{long call} = \text{RangePayoff}.
\]

Operationally:
\begin{itemize}
  \item LP leg captures fee-free and fee-bearing liquidity economics.
  \item Perpetual/call legs isolate directional exposure.
  \item Residual defines the time/range cash-flow primitive.
\end{itemize}

\subsection{Occupation-Time Based Valuation}
The draft proposes two key state variables:
\begin{itemize}
  \item Implied occupation time: $\mathbb{E}[T_{\text{ITM}} \mid \sigma_{\text{impl}}]$
  \item Realized occupation time: $T_{\text{ITM}}(\sigma_{\text{real}})$
\end{itemize}

Business-level pricing hypothesis:
\[
\text{RAN price} \propto \frac{\text{expected fee growth}}{\text{observation frequency}}
\times
f\!\left(\mathbb{E}[T_{\text{ITM}}], T_{\text{ITM}}^{\text{realized}}\right).
\]

\subsection{Replication View}
RAN can be interpreted as a strip of range digital options (cash-or-nothing over observation windows), giving a tractable analytical route once digital prices are available.

\section{Ecosystem}
\subsection{Actors}
\begin{itemize}
  \item LPs: hedge or monetize range/theta exposure.
  \item Borrowers/lenders of LP positions: fund or source structured exposure.
  \item Traders/speculators: trade implied vs realized occupation.
  \item Protocol/market makers: provide RAN secondary liquidity.
\end{itemize}

\subsection{Value Flows}
\begin{itemize}
  \item LP fees feed the accrual leg.
  \item Premiums/funding transfer risk between long and short RAN holders.
  \item Protocol fees are taken on mint/redeem/trade.
\end{itemize}

\subsection{Go-to-Market (MVP)}
\begin{itemize}
  \item Start with one liquid pair and one standardized range schedule.
  \item Offer fixed templates first; add dynamic variants later.
  \item Target active LP managers and vault operators for first adoption.
\end{itemize}

\section{Risks}
\subsection{Model Risk}
\begin{itemize}
  \item Implied occupation may diverge from realized occupation.
  \item Fee-growth forecasts can fail in regime shifts.
\end{itemize}

\subsection{Market and Mechanism Risk}
\begin{itemize}
  \item JIT and microstructure effects may distort expected accrual.
  \item Thin liquidity can break pricing efficiency.
\end{itemize}

\subsection{Technical and Operational Risk}
\begin{itemize}
  \item Smart contract and oracle failure risk.
  \item Settlement-frequency design tradeoffs (precision vs gas).
\end{itemize}

\subsection{Mitigation Direction}
\begin{itemize}
  \item Conservative collateral and liquidation thresholds.
  \item Restricted initial product set and parameter governance.
  \item Formal verification/audits before scale-up.
\end{itemize}

\section{Advantages and Applications}
\subsection{Advantages}
\begin{itemize}
  \item Converts implicit LP theta into explicit tradable instruments.
  \item Improves composability with standard tokenized notes.
  \item Enables direct pricing of JIT externalities.
\end{itemize}

\subsection{Applications}
\begin{itemize}
  \item LP hedging overlays
  \item Range-income structured products
  \item Volatility/occupation-time relative value trades
  \item Protocol-level liquidity quality controls
\end{itemize}

\section{Glossary}
\begin{itemize}
  \item \textbf{RAN}: Range Accrual Note
  \item \textbf{LVR}: Loss-Versus-Rebalancing
  \item \textbf{Occupation time}: time spent by spot within strike interval
  \item \textbf{JIT}: Just-in-time liquidity behavior around execution
\end{itemize}

\section{References}
\begin{itemize}
  \item Panoptic and LP payoff decomposition references
  \item LVR and CLAMM microstructure references
  \item Notes source: \texttt{notes/SCRATCH.md}
\end{itemize}

\end{document}
